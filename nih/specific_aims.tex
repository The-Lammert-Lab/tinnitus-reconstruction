% This example file for NIH submissions was originally written
% by Bruce Donald (http://www.cs.duke.edu/brd/).
% 
% You may freely use, modify and/or distribute this file.
% 
\documentclass[11pt]{nih}
%\documentclass{article}
%\documentclass[12pt]{article}%
% last revision:
\def\mydate{2005-06-09 13:58:03 brd}


%%%%%%% Two column control
\newif\ifdotwocol
\dotwocoltrue   % two col
%\dotwocolfalse   % one col
\long\def\twocol#1#2{\ifdotwocol{#1}\else{#2}\fi}
%%%%%%%

\def\mybeforeequation{\footnotesize}
%\def\mybeforeequation{\small}
%\def\mybeforeequation{}

\def\myafterequation{\renewcommand\baselinestretch{1.1}}
%\def\myafterequation{}

%%%%%%%%%%%%%%%%
%%%%%%%%%%%%%%%%

\def\citeusmark{$^{\textstyle \star}$}
\def\citeus#1#2{\cite{#1}}

\def\crow#1#2{#2}

%\usepackage{denselists}
%\usepackage{scaledfullpage}
\usepackage[dvips]{graphicx}
\usepackage{color}
\usepackage{boxedminipage}
\usepackage{amsfonts}
\usepackage{amsmath}
\usepackage{url}
%\usepackage{times}
%\usepackage{nih}		% PHS 398 Forms
%\usepackage{nihblank}		% For printing on Blank PHS 398 Forms
%\usepackage{confidential}

\def\Paper{grant application}
\def\paper{application}
\def\refappendix{Sec.}

\def\poster{(Poster)}

%Note from brd
\long\def\todo#1{{\bf{To do:}} #1}
%\long\def\todo#1{}
\def\ICRA{IEEE International Conference on Robotics and Automation (ICRA)}

\long\def\squeezable#1{#1}

%\def\a5{$\alpha_{_5}$

\def\a5{5}

%\def\mycaptionsize{\normalsize}
%\def\mycaptionsize{\small}
%\def\mycaptionsize{\small}
\def\mycaptionsize{\footnotesize}
\def\mycodesize{\footnotesize}
\def\myeqnsize{\small}

\def\sheading#1{{\bf #1:}\ }
\def\sheading#1{\subsubsection{#1}}
%\def\sheading#1{\bigskip {\bf #1.}}

\def\ssheading#1{\noindent {\bf #1.}\ } 

\newtheorem{hypothesis}{Hypothesis}
\long\def\hyp#1{\begin{hypothesis} #1 \end{hypothesis}}

\def\cbk#1{[{\em #1}]}

\def\R{\mathbb{R}}
\def\midv{\mathop{\,|\,}}
\def\Fscr{\mathcal{F}}
\def\Gscr{\mathcal{G}}
\def\Sscr{\mathcal{S}}
\def\set#1{{\{#1\}}}
\def\edge{\!\rightarrow\!}
\def\dedge{\!\leftrightarrow\!}
\newcommand{\EOP}{\nolinebreak[1]~~~\hspace*{\fill} $\Box$\vspace*{\parskip}\vspace*{1ex}}
%my way of doing starred references
\newcommand{\mybibitem}[1]{\bibitem{#1} 
\label{mybiblabel:#1}}
\newcommand{\BC}{[}
\newcommand{\EC}{]}
\newcommand{\mycite}[1]{\ref{mybiblabel:#1}\nocite{#1}}
\newcommand{\starcite}[1]{\ref{mybiblabel:#1}\citeusmark\nocite{#1}}


\def\degree{$^\circ$}
\def\R{\mathbb{R}}
\def\Fscr{\mathcal{F}}
\def\set#1{{\{#1\}}}
\def\edge{\!\rightarrow\!}
\def\dedge{\!\leftrightarrow\!}

\long\def\gobble#1{}
\def\Jigsaw{{\sc Jigsaw}}
\def\ahelix{\ensuremath{\alpha}-helix}
\def\ahelices{\ensuremath{\alpha}-helices}
\def\ahelical{$\alpha$-helical}
\def\bstrand{\ensuremath{\beta}-strand}
\def\bstrands{\ensuremath{\beta}-strands}
\def\bsheet{\ensuremath{\beta}-sheet}
\def\bsheets{\ensuremath{\beta}-sheets}
\def\hone{{\ensuremath{^1}\rm{H}}}
\def\htwo{{$^{2}$H}}
\def\cthir{{\ensuremath{^{13}}\rm{C}}}
\def\nfif{{\ensuremath{^{15}}\rm{N}}}
\def\hn{{\rm{H}\ensuremath{^\mathrm{N}}}}
\def\hnone{{\textup{H}\ensuremath{^1_\mathrm{N}}}}
\def\ca{{\rm{C}\ensuremath{^\alpha}}}
\def\catwel{{\ensuremath{^{12}}\rm{C}\ensuremath{^\alpha}}}
\def\ha{{\rm{H}\ensuremath{^\alpha}}}
\def\cb{{\rm{C}\ensuremath{^\beta}}}
\def\hb{{\rm{H}\ensuremath{^\beta}}}
\def\hg{{\rm{H}\ensuremath{^\gamma}}}
\def\dnn{{\ensuremath{d_{\mathrm{NN}}}}}
\def\dan{{\ensuremath{d_{\alpha \mathrm{N}}}}}
\def\jconst{{\ensuremath{^{3}\mathrm{J}_{\mathrm{H}^{\mathrm{N}}\mathrm{H}^{\alpha}}}} }
\def\cbfb{{CBF-$\beta$}}

\newtheorem{defn}{Definition}
\newtheorem{claim}{Claim}

    \gobble{
    \psfrag{CO}[][]{\colorbox{white}{C}}
    \psfrag{OO}[][]{\colorbox{white}{O}}
    \psfrag{CA}[][]{\colorbox{white}{\ca}}
    \psfrag{HA}[][]{\colorbox{white}{\ha}}
    \psfrag{CB}[][]{\colorbox{white}{\cb}}
    \psfrag{HB}[][]{\colorbox{white}{\hb}}
    \psfrag{HN}[][]{\colorbox{white}{\hn}}
    \psfrag{N15}[][]{\colorbox{white}{\nfif}}
    \psfrag{dnn}[][]{\dnn}
    \psfrag{dan}[][]{\dan}
    \psfrag{phi}[][]{$\phi$}
    }

\newenvironment{closeenumerate}{\begin{list}{\arabic{enumi}.}{\topsep=0in\itemsep=0in\parsep=0in\usecounter{enumi}}}{\end{list}}
\def\CR{\hspace{0pt}}           % ``invisible'' space for line break



\newif\ifdbspacing
%\dbspacingtrue  % For double spacing
\dbspacingfalse  % For normal spacing

\ifdbspacing
 \doublespacing
 \newcommand{\capspacing}{\doublespace\mycaptionsize}
\else
 \newcommand{\capspacing}{\mycaptionsize}
\fi

\def\rulefigure{\smallskip\hrule}

% \def\codesize{\normalsize}
\def\codesize{\small}

% Can use macros \be, \ee, \en as shortcuts
%  for \begin{enumerate}, \end{enumerate}, \item
%  respectively.

\def\be{\begin{enumerate}}   % Begin Enumerate
\def\ee{\end{enumerate}}     % End Enumerate
\def\en{\item}               % ENtry (item)
\def\bi{\begin{itemize}}     % Begin Itemize
\def\ei{\end{itemize}}       % End Itemize
\def\bv{\begin{verbatim}}    % Begin Verbatim
\def\ev{\end{verbatim}}      % End Verbatim

\def\matlab{{\sc matlab} }
\def\amber{{\sc amber} }
\def\KS{{$K^*$}}
\def\KSM{{K^*}} % K-Star Math
\def\KSTM{{\tilde{K}^*}}  % K-Star Tilde Math (appx K*)
\def\KOP{{$K^{\dagger}_{o}$}}  % K-Star Optimal partial
\def\KOPM{{K^{\dagger}_{o}}}  % K-Star Optimal partial Math
\def\KP{{$K^{\dagger}$}}  % K-Star partial
\def\KPM{{K^{\dagger}}}  % K-Star partial Math
\def\KTPM{{\tilde{K}^{\dagger}}}  % K-Star Tilde partial Math
\def\KD{{$K_{_D}$}}
\def\KA{{$K_{_A}$}}
\def\qpM{{q_{_P}}}
\def\qlM{{q_{_L}}}
\def\qplM{{q_{_{PL}}}}
\def\qSplM{{q^*_{_{PL}}}}
\def\KSO{{$K^*_{o}$}} % K-Star Optimal
\def\KSOM{{K^*_{o}}}  % K-Star Optimal Math
\def\CBFB{{CBF-$\beta$}}   % Core binding factor beta
\def\argmin{\mathop{\mathrm{argmin}}}
\def\rhl#1{{\em \underline{RYAN}: *\{{#1}\}*}}
\def\set#1{{\left\{ #1 \right\}}}
\def\Escr{{\mathcal{E}}}
\def\Jscr{{\mathcal{J}}}
\def\Kscr{{\mathcal{K}}}
\def\th{{$^{{\mathrm{th}}}$}}

\newtheorem{proposition}{Proposition}
\newtheorem{lemma}{Lemma}

\def\eg{{\emph{e.g.,}}~}
\def\ie{{\emph{i.e.,}}~}

\begin{document}

\bigskip

\appendix 

%\mydate

\setcounter{page}{20} % or whatever

%\noindent{\Large\bf Research Plan}

\section{Specific Aims}

The objective of this proposal is to reveal and analyze cognitive representation of tinnitus percepts
through a novel compressive sensing-based reverse correlation framework.

Tinnitus is the superstitious perception of ringing, buzzing, or hissing in the ears, which can range from mild and transient to debilitating.
In the U.S. alone, it affects 45 million people per year, 30\% of which report their tinnitus to be moderate to severe.
The main treatment is tinnitus retraining therapy (TRT),
a type of habituation therapy that retrains a patient's nervous system by habituating it to non-fictive sounds that mimic the cognitive representation of the tinnitus sound.
TRT requires the medical practitioner to identify an estimate of the tinnitus sound and play it back for the patient.
In practice, this is inexact and inefficient.

Reverse correlation involves the presentation of richly-varying stimuli (\eg white noise) to a subject,
gathering responses of whether the subject perceives in any stimulus a percept corresponding to some abstract, cognitive category (\eg does this sound like your tinnitus?).
Stimuli are random and unbiased, relying on few prior assumptions about the form of the representation.
Regressing observed variables against the stimuli results in inferred cognitive representations.
Indeed, the number of features
recoverable using reverse correlation is effectively unlimited, and only constrained by broad
assumptions of the method (\eg linearity). Higher-dimensional characterization of perceptual
representations may help resolve many ambiguities inherent to lower-dimensional characterizations by
supplying additional information necessary to distinguish between representations from different subjects.

We have constructed a data processing pipeline using reverse correlation to infer unbiased estimates of cognitive representations of visual stimuli (\eg letters, faces) and phonological sounds (\eg /ee/).
In the proposed work, we aim to directly uncover and reconstruct estimates of cognitive representations of tinnitus percepts using reverse correlation,
improve the efficiency of reverse correlation through signal reconstruction,
and characterize cognitive representations of tinnitus for use in diagnosis and treatment.
This will result in an open-source software tool that efficiently constructs unbiased cognitive representations of tinnitus percepts
and insight into subject-to-subject and longitudinal variability in tinnitus self-experience to improve diagnosis and treatment.

Our long-term goal is to efficiently uncover effective representations of tinnitus percepts to inform diagnosis and treatment
and to explain variability in tinnitus experience.
Our guiding hypothesis is that reverse correlation will produce individualized, unbiased, high-dimensional cognitive representations of tinnitus percepts
and that we can optimize reverse correlation to require fewer samples using signal reconstruction techniques.
[RATIONALE]
We will test our central hypothesis and attain our objective via the following specific aims:

\subsection{Uncover and reconstruct cognitive representations of tinnitus}
Participants will classify white noise stimuli as sounding like their tinnitus or not.
We will use reverse correlation to infer an unbiased estimate of each particpant's cognitive representation of their tinnitus percept.

\subsection{Develop efficient reconstruction algorithm for cognitive representations using compressive sensing}
We will develop an open-source compressive sensing data analysis pipeline to acquire convergent cognitive representations using fewer trials.
Our pilot data demonstrates that compressive sensing affords a 90\% decrease in number of trials for equivalent resulant latent representation.
We will also develop and distribute an open-source software
implementation the of the reverse correlation method for speech perception, including the proposed
compressive sensing framework, for use by the research community.

\subsection{Characterize and interpret cognitive representations of tinnitus}
Our compressive sensing framework will allow for rapid data acquisition from participants.
We will characterize cognitive representations of tinnitus between participants and over time.
What is the space of tinnitus cognitive representations? Are there subtypes or clusters on a lower dimensional manifold?
We will employ dimensionality reduction and clustering algorithms to represent and visualize the data,
providing direct insight into commonalities and variability of tinnitus experience in and between subjects.

\end{document}

