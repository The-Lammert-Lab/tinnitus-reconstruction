\section{Project Description}

\subsection{Introduction}

Current attempts to characterize perceptual mechanism are limited in scope due to inefficiencies of a primary method used to characterize neural tuning, \emph{reverse correlation}.
Reverse correlation allows for unconstrained and unbiased estimation of latent neural representations using fairly simple stimulus-response data representations,
including psychophysical kernels that drive the top-down processes of perception (\eg face or phoneme recognition)
and even abstract psychological categories (\eg ``male'' vs ``female'' faces).
The method has broad applicability for characterizing any aspect of neurological, cognitive, or psychological function that can be modeled as a transductive process.

In reverse correlation, stimulus-response data are elicited via the presentation of richly-varying stimuli (\eg white noise).
Latent representations inherent in the system of interest can be estimated by regressing observed responses against the stimuli over many trials.
However, the number of stimulus-response samples required for accurate estimation is typically very large.
This inefficiency limits the feasibility of applying reverse correlation to only those experimental protocols that are very stable, such as \emph{in-vivo} preparations or where subject participation can be maintains over long timelines (\ie two weeks).
To circumvent this limitation, studies often (a) limit the richness of the stimuli, or (b) impose some constraints on the inferred representations,
for instance, by smoothing the raw estimates.
The constraints imposed by these approaches ultimately inhibit the full expressiveness of reverse correlation, leading to estiamtes that are biased.

We propose to develop an advanced signal processing pipeline based on \emph{compressive sensing} that will enable the research community to overcome the inefficiencies of existing reverse correlation methods.
Compressive sensing is a signal processing technique that can dramatically improve the efficiency of traditional sampling and signal estimation methods.
The technique has recently gained wide recognition in domains such as medical imaging,
where considerations of efficiency and bias reduction are critical.
Compressive sensing holds promise to similarly improve the efficiency of reverse correlation,
without the drawbacks of biasing estimates.
This will extend the range of experimental protocols that are feasible
as well as the range of perceptual mechanisms that can be estimated in correspondence with this technical improvements.
Moreover, compressive sensing can be directly substituted for conventional, regression-based estimation
with no other required changes to existing experimental protocols.

Latent representations, like many signals of interest, are compressible,
meaning that they can be represented as a linear combination of a small number of functions from an appropriately-selected basis set
(\ie $s = \Psi^T x$, for weights $s$, basis $\Psi$, and signal $x$).
A key insight of compressive sensing is that,
if one assumes that the responses stem from a process of comparing stimuli to the latent representation
(\ie $y = \Phi x$, for responses $y$ and measurement operation $\Phi$),
then it becomes possible to estimate the latent representation using only a small number of measurements
by acquiring the basis function representation directly
(\ie $y = \Phi \Psi s$) via sparse optimization approaches (to find $s$).
The responses can be continuous (\eg neural firing rates), ordinal similarity scores,
or binary (\ie yes/no).

\subsection{Proposed Study}

The objective of this work is to develop and validate a compressive sensing data processing pipeline,
culminating in an open-source software tool, that will allow for efficient and accurate reconstructions of latent representations using data obtained via the reverse correlation method.

\paragraph{Specific Aim \#1}

We will validate the proposed use of compressive sensing for relevant types of neural, cognitive, and behavioral data.
Validation will be accomplised through simulation studies, analysis of existing databases (\eg those available in \emph{Dryad}),
and through the collection of novel psychophysical data.
Outcomes will be assessed by examining estimation performance as a function of sample size,
including (a) reconstruction accuracy with few samples, (b) gains in high-sample reconstruction accuracy,
and (c) minimal sample size to achieve high-end convergent accuracy.
We will determine the optimal parameters for compressive sensing, which we expect to depend on the latent representation of interest, stimulus type, and response noise.
Considered parameters will include appropriate input/output representations, reconstruction algorithm, and basis type/sparsity.

\paragraph{Specific Aim \#2}

We will develop and distribute an open-source software implementation of the compressive sensing framework,
informed by the findings above, for use by the research community.
In order to make the platform maximally accessible, interfaces will be developed for widely-used, high-level programming languages
(including MATLAB, Python, R, and Julia) in addition to command-line executables.
Making the platform open-source provides opportunities for the community to better understand, customize, and improve the software for their own specific needs. \\ \\
\noindent
In sum, reverse correlation has the potential to uncover latent representations underlying
perception, and transform our understanding perceptual mechanisms at various levels of
investigation: neural, cognitive and psychological. However, in order for this potential to be fully
realized, the fundamental inefficiency of reverse correlation paradigms must be overcome.
Compressive sensing holds promise to overcome this limitation by dramatically improving the
efficiency of reverse correlation, enabling its extension to perceptual mechanisms that are out of
reach using current methods. The work proposed here will enable researchers to access the
promise of compressive sensing for broadening the impact of reverse correlation.

% The Project Description should provide a clear statement of the work to be undertaken and must include:
% objectives for the period of the proposed work and expected significance; relation to longer-term goals of the PI's
% project; and relation to the present state of knowledge in the field, to work in progress by the PI under other
% support and to work in progress elsewhere.

% The Project Description should outline the general plan of work, including the broad design of activities to be
% undertaken, and, where appropriate, provide a clear description of experimental methods and procedures.
% Proposers should address what they want to do, why they want to do it, how they plan to do it, how they will
% know if they succeed, and what benefits could accrue if the project is successful. The project activities may be
% based on previously established and/or innovative methods and approaches, but in either case must be well
% justified. These issues apply to both the technical aspects of the proposal and the way in which the project may
% make broader contributions.

\subsection{Broader Impacts of the Proposed Work}
The Project Description must contain, as a separate section within the narrative, a section labeled ``Broader
Impacts of the Proposed Work". This section should provide a discussion of the broader impacts of the proposed
activities. Broader impacts may be accomplished through the research itself, through the activities that are
directly related to specific research projects, or through activities that are supported by, but are complementary to 
the project. NSF values the advancement of scientific knowledge and activities that contribute to the
achievement of societally relevant outcomes. Such outcomes include, but are not limited to: full
participation of women, persons with disabilities, and underrepresented minorities in science, technology, engineering, and
mathematics (STEM); improved STEM education and educator development at any level; increased public
scientific literacy and public engagement with science and technology; improved well-being of individuals in
society; development of a diverse,globally competitive STEM workforce; increased partnerships between
academia, industry, and others; improved national security; increased economic competitiveness of the United
States; and enhanced infrastructure for research and education.

\subsection{Results from Prior NSF Support}
If any PI or co-PI identified on the project has received NSF funding (including any current
funding) in the past five years, in formation on the award(s) is required,
irrespective of whether the support was directly related to the proposal or not.
In cases where the PI or co-PI has received more than one award (excluding amendments),
they need only report on the one award most closely related to the proposal. Funding includes not just salary
support, but any funding awarded by NSF. The following information must be provided:\\

\noindent
\emph{\underline{Name of PI}}: NSF-Program (Award Number) ``Title of the Project'' (\$AMOUNT, PERIOD OF SUPPORT). 
{\bf Publications:} List of publications resulting from the NSF award. A complete bibliographic citation for each
publication must be provided either in this section or in the References Cited section of the proposal); if
none, state: ``No publications were produced under this award.'' {\bf Research Products:} evidence of research products 
and their availability, including, but not limited to: data, publications, samples, physical collections, software, 
and models, as described in any Data Management Plan.