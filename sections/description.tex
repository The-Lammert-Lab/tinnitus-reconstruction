\section{Project Description}
\subsection{Introduction}
\subsection{Proposed Study}
The Project Description should provide a clear statement of the work to be undertaken and must include:
objectives for the period of the proposed work and expected significance; relation to longer-term goals of the PI's
project; and relation to the present state of knowledge in the field, to work in progress by the PI under other
support and to work in progress elsewhere.

The Project Description should outline the general plan of work, including the broad design of activities to be
undertaken, and, where appropriate, provide a clear description of experimental methods and procedures.
Proposers should address what they want to do, why they want to do it, how they plan to do it, how they will
know if they succeed, and what benefits could accrue if the project is successful. The project activities may be
based on previously established and/or innovative methods and approaches, but in either case must be well
justified. These issues apply to both the technical aspects of the proposal and the way in which the project may
make broader contributions.

\subsection{Broader Impacts of the Proposed Work}
The Project Description must contain, as a separate section within the narrative, a section labeled ``Broader
Impacts of the Proposed Work". This section should provide a discussion of the broader impacts of the proposed
activities. Broader impacts may be accomplished through the research itself, through the activities that are
directly related to specific research projects, or through activities that are supported by, but are complementary to 
the project. NSF values the advancement of scientific knowledge and activities that contribute to the
achievement of societally relevant outcomes. Such outcomes include, but are not limited to: full
participation of women, persons with disabilities, and underrepresented minorities in science, technology, engineering, and
mathematics (STEM); improved STEM education and educator development at any level; increased public
scientific literacy and public engagement with science and technology; improved well-being of individuals in
society; development of a diverse,globally competitive STEM workforce; increased partnerships between
academia, industry, and others; improved national security; increased economic competitiveness of the United
States; and enhanced infrastructure for research and education.

\subsection{Results from Prior NSF Support}
If any PI or co-PI identified on the project has received NSF funding (including any current
funding) in the past five years, in formation on the award(s) is required,
irrespective of whether the support was directly related to the proposal or not.
In cases where the PI or co-PI has received more than one award (excluding amendments),
they need only report on the one award most closely related to the proposal. Funding includes not just salary
support, but any funding awarded by NSF. The following information must be provided:\\

\noindent
\emph{\underline{Name of PI}}: NSF-Program (Award Number) ``Title of the Project'' (\$AMOUNT, PERIOD OF SUPPORT). 
{\bf Publications:} List of publications resulting from the NSF award. A complete bibliographic citation for each
publication must be provided either in this section or in the References Cited section of the proposal); if
none, state: ``No publications were produced under this award.'' {\bf Research Products:} evidence of research products 
and their availability, including, but not limited to: data, publications, samples, physical collections, software, 
and models, as described in any Data Management Plan.
